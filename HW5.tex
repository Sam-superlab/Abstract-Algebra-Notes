\documentclass[12pt,letterpaper, onecolumn]{exam}
\usepackage{amsmath}
\usepackage{tikz}
\usepackage{tikz-cd}
\usetikzlibrary{positioning}
\usepackage{amsthm, amsfonts}
\usepackage{mathrsfs}

% Number equations by question (nice for homework)
\numberwithin{equation}{question}

% Theorem-like environments, numbered per question
\newtheorem{theorem}{Theorem}[question]
\newtheorem{lemma}[theorem]{Lemma}
\theoremstyle{definition}
\newtheorem{definition}[theorem]{Definition}
\theoremstyle{remark}
\newtheorem{remark}{Remark}

% Make the QED symbol a solid square (optional)
\renewcommand{\qedsymbol}{$\blacksquare$}
\newcommand{\Q}[0]{\mathbb{Q}}
\newcommand{\N}[0]{\mathbb{N}}
\newcommand{\Z}[0]{\mathbb{Z}}
\newcommand{\R}[0]{\mathbb{R}}
\newcommand{\G}[0]{\mathscr{G}}
\newcommand{\la}[0]{\langle}
\renewcommand{\O}[0]{\mathcal{O}}
\newcommand{\ra}{\rangle}
\newcommand{\mydistance}{.6cm}
\usepackage{amssymb}
\usepackage[lmargin=71pt, tmargin=1.2in]{geometry}  %For centering solution box
\lhead{ }
\rhead{ }
% \chead{\hline} % Un-comment to draw line below header
\thispagestyle{empty}   %For removing header/footer from page 1

\begin{document}

\begingroup  
    \centering
    \LARGE Abstract Algebra\\
    \LARGE Problem Set 5\\[0.5em]
    \large \today\\[0.5em]
    \large Xuyi (Sam) Ren\par
    % \large Roll Number\par
    \large MAT-321\par
\endgroup
\rule{\textwidth}{0.4pt}
\pointsdroppedatright   %Self-explanatory
\printanswers
\renewcommand{\solutiontitle}{\noindent\textbf{Ans:}\enspace}   %Replace "Ans:" with starting keyword in solution box
\begin{questions}


\question[8.8] Compute $\sigma^{100}$ where
\[
\sigma = \begin{pmatrix}
1 & 2 & 3 & 4 & 5 & 6 \\
3 & 1 & 4 & 5 & 6 & 2
\end{pmatrix}.
\]

\begin{solution}
    We determine the orbit of the element 1 by repeatedly applying $\sigma$:

    \begin{equation*}
        1 \xrightarrow{\sigma} 3 \xrightarrow{\sigma} 4 \xrightarrow{\sigma} 5 \xrightarrow{\sigma} 6 \xrightarrow{\sigma} 2 \xrightarrow{\sigma} 1
    \end{equation*}
    
    The orbit of 1 is $\{1, 3, 4, 5, 6, 2\}$.  And therefore, $\sigma^6 = e$.

    We use the division algorithm to reduce the exponent $100$ modulo 6:
    \[ 100 = 6\times16+ 4\]
    Substituting this into the expression for $\sigma^{100}$:
    \[ \sigma^{100} = \sigma^{6(16) + 4} = (\sigma^6)^{16}\sigma^4 = e^{16}\sigma^4 = \sigma^4. \]

    We compute $\sigma^4$ by showing an example computation for a single element, and the rest will be similar procedure:
    \begin{itemize}
        \item $1 \xrightarrow{\sigma} 3 \xrightarrow{\sigma} 4 \xrightarrow{\sigma} 5 \xrightarrow{\sigma} 6$. So $\sigma^4(1) = 6$.
    \end{itemize}
Then we shall write it in a matrix notation by our calculation results:
    \[ \sigma^{100} = \begin{pmatrix}
    1 & 2 & 3 & 4 & 5 & 6 \\
    6 & 5 & 2 & 1 & 3 & 4
    \end{pmatrix}. \]
\end{solution}





\question[8.12]Let $A$ be a set and let $\sigma \in S_A$. For a fixed $a \in A$, the set
\[
\mathcal{O}_{a,\sigma} = \{\sigma^n(a) \mid n \in \mathbb{Z}\}
\]
is the \textbf{orbit} of $a$ under $\sigma$. In Exercises 11 through 13, find the
orbit of $1$ under the permutation defined prior to Exercise~1.

$\tau$
\begin{solution}
    We consider the permutation $\tau$ defined as:
    \[
    \tau =
    \begin{pmatrix}
    1 & 2 & 3 & 4 & 5 & 6 \\
    2 & 4 & 1 & 3 & 6 & 5
    \end{pmatrix}
    \]
    We wish to find the orbit of $1$ under $\tau$, denoted as $\mathcal{O}_{1,\tau} = \{ \tau^n(1) \mid n \in \mathbb{Z} \}$. To do this, we calculate the sequence of images $\tau^n(1)$ for $n = 0, 1, 2, \dots$ until the values repeat.

    \begin{itemize}
        \item For $n=0$, by definition $\tau^0$ is the identity map, so $\tau^0(1) = 1$.
        \item For $n=1$, reading from the matrix definition, we have $\tau(1) = 2$.
        \item For $n=2$, we apply $\tau$ to the previous result:
        \[ \tau^2(1) = \tau(\tau(1)) = \tau(2) = 4. \]
        \item For $n=3$, we apply $\tau$ again:
        \[ \tau^3(1) = \tau(\tau^2(1)) = \tau(4) = 3. \]
        \item For $n=4$, we have:
        \[ \tau^4(1) = \tau(\tau^3(1)) = \tau(3) = 1. \]
    \end{itemize}

    Since $\tau^4(1) = 1$, the sequence of values repeats starting from $n=4$. The distinct elements generated by these powers are $\{1, 2, 4, 3\}$.
    
    Thus, the orbit of 1 under $\tau$ is:
    \[ \mathcal{O}_{1,\tau} = \{1, 2, 3, 4\}. \]
\end{solution}

\question[8.18]
\begin{enumerate}
\item[18.] Consider the group $S_3$ of Example~8.7.
\begin{enumerate}
\item[(a)] Find the cyclic subgroups $\langle \rho_1 \rangle$, $\langle \rho_2 \rangle$, and $\langle \mu_1 \rangle$ of $S_3$.
\item[(b)] Find \emph{all} subgroups, proper and improper, of $S_3$ and give the subgroup diagram for them.
\end{enumerate}
\end{enumerate}


\begin{solution}
\textbf{(a) }
By Table 8.8 in textbook
\begin{itemize}
    \item Since the table shows \(\mu_1\mu_1=\rho_0\), we have
    \[
        \langle \mu_1\rangle=\{\rho_0,\mu_1\}.
    \]
    \item Since the table shows \(\rho_1\rho_1=\rho_2\) and \(\rho_1\rho_2=\rho_0\), we get
    \[
        \langle \rho_1\rangle=\{\rho_0,\rho_1,\rho_2\}.
    \]
    \item Since the table shows \(\rho_2\rho_2=\rho_1\) and \(\rho_2\rho_1=\rho_0\), we get
    \[
        \langle \rho_2\rangle=\{\rho_0,\rho_2,\rho_1\}=\{\rho_0,\rho_1,\rho_2\}.
    \]
\end{itemize}
So \(\langle\rho_1\rangle=\langle\rho_2\rangle\).

\medskip
\textbf{(b)}

Let \(H\le S_3\). Then \(H\) is nonempty and closed under multiplication and inverses.

\medskip
Identity:
Pick any \(h\in H\). Since \(H\) is a subgroup, \(h^{-1}\in H\) and thus
\(hh^{-1}=\rho_0\in H\).

\medskip
If any rotation is in $H$ then the rest two are also in, thus, $\la \rho_0 \ra \subseteq H$. 

If \(\rho_1\in H\), then by closure and the table,
\[
\rho_1\rho_1=\rho_2 \in H,\qquad \rho_1\rho_2=\rho_0\in H,
\]
so \(\{\rho_0,\rho_1,\rho_2\}\subseteq H\). and $\la \rho_1\ra=\la \rho_2\ra=\la \rho_0\ra$

\medskip
Subgroups with rotation and reflections:

Assume \(\{\rho_0,\rho_1,\rho_2\}\subseteq H\) and also \(\mu_1\in H\).
Then the table gives
\[
\rho_1\mu_1=\mu_3 \in H,\qquad \rho_2\mu_1=\mu_2\in H.
\]
So all \(\mu_1,\mu_2,\mu_3\) are in \(H\), hence \(H\) contains all six elements:
\[
H=\{\rho_0,\rho_1,\rho_2,\mu_1,\mu_2,\mu_3\}=S_3.
\]
The same argument works for \(\mu_2\) or \(\mu_3\).

Therefore, if \(H\) contains a nontrivial rotation and any \(\mu_i\), then \(H=S_3\).
Otherwise the only possibility in this case is
\[
H=\{\rho_0,\rho_1,\rho_2\}.
\]

\medskip
We claim that If \(H\) do not contains $\rho_1,\rho_2$, equivalently, all rotations except the identity, then it can contain at most one reflection.

Suppose \(\rho_1,\rho_2\notin H\). If \(H\) contains \(\mu_1\), then closure gives \(\mu_1\mu_1=\rho_0\),
so \(\{\rho_0,\mu_1\}\subseteq H\).

If \(H\) also contained a different reflection, say \(\mu_2\), then the table shows
\[
\mu_1\mu_2=\rho_1,
\]
which would force \(\rho_1\in H\), contradicting the assumption that \(H\) has no nontrivial rotations.
Therefore, in such a case, \(H\) is either \(\{\rho_0\}\) or one of
\(\{\rho_0,\mu_1\},\{\rho_0,\mu_2\},\{\rho_0,\mu_3\}\)

\medskip
Therefore, the full list of subgroups of \(S_3\) is
\[
\{\rho_0\},\quad
\{\rho_0,\mu_1\},\ \{\rho_0,\mu_2\},\ \{\rho_0,\mu_3\},\quad
\{\rho_0,\rho_1,\rho_2\},\quad
S_3.
\]

\medskip

% Requires: \usepackage{tikz} and \usetikzlibrary{positioning}
% Preamble:
% \usepackage{tikz}
% \usetikzlibrary{positioning}

\begin{center}
\begin{tikzpicture}[
  node distance=16mm and 20mm, % (vertical) and (horizontal)
  box/.style={draw, rounded corners, align=center, inner sep=2pt, text width=2cm},
  small/.style={draw, rounded corners, align=center, inner sep=3pt},
  line width=0.45pt
]
  \node[small] (G) {$S_3$};

  % Middle layer (wide boxes)
  \node[box] (R)  [below=of G] {$\langle \rho_1\rangle=\langle \rho_2\rangle$\\[-1pt]
    {\footnotesize $\{\rho_0,\rho_1,\rho_2\}$}};
  \node[box] (M1) [left=of R] {$\langle \mu_1\rangle$\\[-1pt]
    {\footnotesize $\{\rho_0,\mu_1\}$}};
  \node[box] (M2) [right=of R] {$\langle \mu_2\rangle$\\[-1pt]
    {\footnotesize $\{\rho_0,\mu_2\}$}};
  \node[box] (M3) [right=of M2] {$\langle \mu_3\rangle$\\[-1pt]
    {\footnotesize $\{\rho_0,\mu_3\}$}};

  % Bottom
  \node[small] (E) [below=22mm of R] {$\{\rho_0\}$};

  % Edges
  \foreach \H in {M1,R,M2,M3}{
    \draw (G) -- (\H);
    \draw (\H) -- (E);
  }
\end{tikzpicture}
\end{center}


\end{solution}


\question[8.42]
Let $B\subseteq A$. 
Determine whether the set
\[
H = \{\sigma \in S_A \mid \sigma[B] \subseteq B\}
\]
is a subgroup of $S_A$ under composition.


\begin{solution}
We determine whether
\[
H=\{\sigma\in S_A\mid \sigma[B]\subseteq B\}
\]
is a subgroup of $S_A$.

We show that $H$ is {not} a subgroup in general by provide a counterexample.


Let $f:\mathbb{Z}\to A$ be a bijection and define $a_n:=f(n)$ for each $n\in\mathbb{Z}$.
\medskip
Let $A=\{a_n\mid n\in\mathbb{Z}\}$ be a set whose elements are labeled by integers.
Let
\[
B=\{a_n\mid n\ge 0\}\subseteq A.
\]
Define a permutation $\tau\in S_A$ by
\[
\tau(a_n)=a_{n+1}\qquad \text{for all } n\in\mathbb{Z}.
\]
This map is bijective because it has an inverse map $\tau^{-1}:A\to A$ given by
\[
\tau^{-1}(a_n)=a_{n-1}\qquad \text{for all } n\in\mathbb{Z}.
\]
Hence $\tau\in S_A$.

Next, compute the image of $B$:
\[
\tau[B]=\{\tau(a_n)\mid n\ge 0\}=\{a_{n+1}\mid n\ge 0\}=\{a_m\mid m\ge 1\}\subseteq \{a_m\mid m\ge 0\}=B.
\]
Therefore $\tau\in H$.

However,
\[
\tau^{-1}(a_0)=a_{-1}\notin B,
\]
so $\tau^{-1}[B]\nsubseteq B$, which means $\tau^{-1}\notin H$.

Thus $H$ is not closed under inverses, and therefore $H$ is not a subgroup of $S_A$.
\end{solution}






\question[8.48] Prove that if $\mathcal{O}_{a,\sigma}$ shares an element in common with $\mathcal{O}_{b,\sigma}$, then $\mathcal{O}_{a,\sigma} = \mathcal{O}_{b,\sigma}$.

\begin{solution}
    Let $x$ be an element common to both orbits. By the definition of an orbit, there exist integers $n, m \in \mathbb{Z}$ such that:
    \[ x = \sigma^n(a) \quad \text{and} \quad x = \sigma^m(b). \]
    Equating these expressions, we have:
    \[ \sigma^n(a) = \sigma^m(b). \]
    
    Since $\sigma\in S_A$, and by group axiom $\G3$ the inverse exists. We apply the inverse permutation $\sigma^{-m}$ to both sides to isolate $b$:
    \[ \sigma^{-m}(\sigma^n(a)) = \sigma^{-m}(\sigma^m(b)) \implies \sigma^{n-m}(a) = b. \]
    Since $k = n-m$ is an integer, this shows that $b \in \mathcal{O}_{a,\sigma}$.
    
    Now we show the sets are equal by double containment. Let $y \in \mathcal{O}_{b,\sigma}$ be arbitrary. Then $y = \sigma^j(b)$ for some $j \in \mathbb{Z}$.
    Substituting $b = \sigma^k(a)$ into this expression:
    \[ y = \sigma^j(\sigma^k(a)) = \sigma^{j+k}(a). \]
    Since $j+k \in \mathbb{Z}$, it follows that $y \in \mathcal{O}_{a,\sigma}$.
    Thus, $\mathcal{O}_{b,\sigma} \subseteq \mathcal{O}_{a,\sigma}$.
    
    By symmetry, we can solve for $a$ ($a = \sigma^{m-n}(b)$) and use the same logic to show $\mathcal{O}_{a,\sigma} \subseteq \mathcal{O}_{b,\sigma}$.
    
    Therefore, $\mathcal{O}_{a,\sigma} = \mathcal{O}_{b,\sigma}$.
\end{solution}










\question[] collaboration comments



\section{Appendix}

\end{questions}

\end{document}
