\documentclass[10pt]{article}

% ---------- Page Layout ----------
\usepackage[margin=0.7in]{geometry}
\usepackage{multicol}
\usepackage[most]{tcolorbox}
\usepackage{pagecolor}

\setlength{\columnsep}{0.4in}
\pagecolor{lightgray}
% ---------- Math Packages ----------
\usepackage{amsmath, amssymb, amsthm}
\usepackage{mathrsfs}

% ---------- Formatting ----------
\usepackage{enumitem}
\setlist[itemize]{leftmargin=1em, nosep}
\setlist[enumerate]{leftmargin=1em, nosep}

\setlength{\parindent}{0pt}
\setlength{\parskip}{4pt}

% ---------- Custom Environments ----------
\newtheoremstyle{defstyle}
  {4pt} % space above
  {4pt} % space below
  {}    % body font
  {}    % indent
  {\bfseries} % theorem head font
  {.}   % punctuation
  {0.5em} % space after head
  {}

  \tcolorboxenvironment{definition}{
  enhanced,
  breakable,
  colback=white,              % inner (definition) background
  colframe=gray!60,           % border color
  colbacktitle=gray!20,       % title background
  arc=3mm,                    % rounded corners
  boxrule=0.8pt,              % border thickness
  left=6pt,
  right=6pt,
  top=6pt,
  bottom=6pt,
  fonttitle=\bfseries,
}


\theoremstyle{defstyle}
\newtheorem{definition}{Definition}

% ---------- Shortcuts ----------
\newcommand{\R}{\mathbb{R}}
\newcommand{\N}{\mathbb{N}}
\newcommand{\Z}{\mathbb{Z}}
\newcommand{\Q}{\mathbb{Q}}
\newcommand{\G}{\mathscr{G}}
\newcommand{\bij}{\leftrightarrow}

% ---------- Document ----------
\begin{document}

\begin{center}
    {\Large \textbf{Definition Sheet}}\\
    \vspace{0.2em}
    {\small Abstract Algebra / First chapter \quad • \quad Xuyi Ren \quad • \quad \today }
\end{center}

\vspace{0.5em}

\begin{multicols}{2}



\begin{definition}[\textbf{Binary Operation}]
    A \textbf{Binary Operation *} on set $S$ is a function $*:S\times S\to S$, For each $(a,b)\in S\times S$ we denote $*((a,b))$ as $a*b$
\end{definition}

\begin{definition}[\textbf{Closure}]
    Let $*$ be an binary operation on $S$, let $H\subseteq S$ be a subset of $S$. Then the subset $H$ is called \textbf{closed} under $*$ if for all $a,b\in H$ we have $a*b\in H$. 
\end{definition}

\begin{definition}[\textbf{Induced operation}]
    In the same setup of closure, the binary operation on $H$ given by restricting $*$ to set $H$ is called induced operation of $*$ on $H$
\end{definition}

\begin{definition}[\textbf{commutative}]
    An binary operation $*$ on set $S$ is called commutative iff for all $a,b\in S$ we have $a*b=b*a$
\end{definition}

\begin{definition}[\textbf{Associative}]
    A Binary operation $*$ on set $S$ is called associative iff for all $a,b,c\in S$ we have $a*(b*c)=(a*b)*c$
\end{definition}

\begin{definition}[\textbf{Binary Algebraic Structure}]
    A Binary Algebraic Structure $<S,*>$ is a set $S$ together with a binary operation $*$ on $S$.
\end{definition}

\begin{definition}[\textbf{Isomorphism}]
    Let $<S,*>$ and $<S',*'>$ be two binary algebraic structure, an isomorphism $\phi$ is an bijection between $S$ and $S'$ such that 
    \begin{equation*}
        \phi(x*y)=\phi(x)*'\phi(y)
    \end{equation*}
    for all $x,y\in S$. If such a mapping exists, then $S$ and $S'$ are called \textbf{isomorphic binary structures}.
\end{definition}

\begin{definition}[\textbf{Structural property}]
    A structural property of a binary structure is one that must be shared by any isomorphic structure.
\end{definition}

\begin{definition}[\textbf{Identity}]
    Let $<S,*>$ be an binary algebraic structure, An element $e$ of $S$ is an identity element for $*$ if 
    \begin{equation*}
        a*e=e*a=a
    \end{equation*}
    for all $a\in S$.
\end{definition}

\textbf{Weekly 1 Quiz Done Here}

\section{Group Theory}

\begin{definition}[Group]
A group $(G,*)$ is set that is closed under binary operation $*$  with associativity, identity, and inverses 
\end{definition}

\begin{definition}[\textbf{Abelian Group}]
    A group $(G,\cdot)$ is abelian if $\cdot$ is commutative
\end{definition}

\begin{definition}[\textbf{General Linear Group of degree n}]
    A General linear group of degree $n$ is a set of $n\times n$ invertible matrix with matrix multiplication defined on it.
\end{definition}

\begin{definition}[\textbf{Order of Group}]
If $G$ is a group, then the \emph{order} $|G|$ of $G$ is the number of elements in $G$.
(Recall from Section~0 that, for any set $S$, $|S|$ is the cardinality of $S$.)
\end{definition}

\begin{definition}[\textbf{Subgroup}]
If a subset $H$ of a group $G$ is closed under the binary operation of $G$ and if $H$ with the
induced operation from $G$ is itself a group, then $H$ is a \emph{subgroup} of $G$.
We shall let $H \le G$ or $G \ge H$ denote that $H$ is a subgroup of $G$, and
$H < G$ or $G > H$ shall mean $H \le G$ but $H \ne G$.
\end{definition}


\begin{definition}[\textbf{Improper, Proper and trivial subgroup}]
If $G$ is a group, then the subgroup consisting of $G$ itself is the \emph{improper subgroup} of $G$.
All other subgroups are \emph{proper subgroups}.
The subgroup $\{e\}$ is the \emph{trivial subgroup} of $G$.
All other subgroups are \emph{nontrivial}.
\end{definition}

\begin{definition}[\textbf{Cyclic subgroup of $G$ generate by $a$}]
Let $G$ be a group and let $a \in G$. Then the subgroup
$\{a^n \mid n \in \mathbb{Z}\}$ of $G$, characterized in Theorem~5.17,
is called the \emph{cyclic subgroup} of $G$ generated by $a$, and denoted
by $\langle a \rangle$.
\end{definition}

\begin{definition}[\textbf{Generator}]
An element $a$ of a group $G$ \emph{generates} $G$ and is a \emph{generator}
for $G$ if $\langle a \rangle = G$.
A group $G$ is \emph{cyclic} if there is some element $a \in G$
that generates $G$.
\end{definition}

\begin{definition}[\textbf{Cyclic Group}]
A group $G$ is called \emph{cyclic} if there exists an element $a \in G$
such that
\[
G = \langle a \rangle = \{ a^n \mid n \in \mathbb{Z} \}.
\]
\end{definition}

\begin{definition}[\textbf{Order of an element in group}]
  Let $a$ be an element of a group $G$. If the cyclic subgroup $\langle a \rangle$ of $G$ is finite then the \textbf{order of} $a$ is the order $|\langle a\rangle|$ of this cyclic subgroup.  
\end{definition}

\begin{definition}[\textbf{Greatest common divisor}]
Let $r,s$ be two positive integer, the positive generator $d$ is of cyclic group:
\begin{equation*}
    H=\{nr+ms|n,m\in \Z\}
\end{equation*}
    under addition is the \textbf{Greatest common divisor} of $r$ and $s$. We write $d=\gcd(r,s)$
\end{definition}


\begin{definition}[\textbf{Relative Prime}]
Two integer are relative prime if their gcd equals 1    
\end{definition}


\begin{definition}[\textbf{Finitely Generated}]

Let $G$ be a group and let $a_i\in G$ for $i\in I$. The smallest subgroup of $G$ containing $\{a_i|i\in I\}$ is the \textbf{subgroup generated by }$\{a_i|i\in I\}$. If this subgroup is all of $G$, then $a_i$ \textbf{generates} $G$ and $a_i$ is a \textbf{generator} of $G$. If there is a finite set $\{a_i|i\in I\}$ that generates $G$ then $G$ is finitely generated.
    
\end{definition}
\textbf{Week 2 Quiz Done Here}

\subsection{Permutation Groups}

\begin{definition}[\textbf{Permutation}]
    A \textbf{permutation} on a set $A$ is a bijection from $A$ to $A$.
\end{definition}


\begin{definition}[\textbf{Symmetric group on $n$ letters}]
    Let $A:=\{1,2,3,...,n\}$ be finite, the group of all permutations of $A$ is the symmetric group on $n$ letters, denoted by $S_n$. In particular, $|S_n|=n!$
 \end{definition}

\begin{definition}[$n$th dihedral group]
    The $n$-th dihedral group $D_n$ is the group of symmetries of the regular $n-gon$. 
\end{definition}

\subsubsection{Cayley's Theorem}

\begin{definition}[\textbf{image of $H$ under $f$}]
    Let $f: A\to B$ be a function and let $H\subseteq A$.  Then the \textbf{image of $H$ under $f$ } is $\{f(h)|h\in H\}$ and denoted by $f[H]$.
\end{definition}

\subsection{Orbits, Cycles, and the Alternating Groups}

\subsubsection{Orbits}

\begin{definition}[\textbf{Orbits}]
Let $\sigma:A\bij A$, then the equivalence class $[a] := \{\, b \in A \mid \exists n \in \mathbb{Z},\ \sigma^n(a)=b \,\}
$ are the orbits of $\sigma$. $Orb_\sigma(a)=[a]$. 
    
\end{definition}

\begin{definition}[\textbf{Cycle, Length of cycle}]

A permutation $\sigma\in S_n$ is a cycle if it has at most one orbit containing more than one element. The \textbf{length} of a cycle is the number of elements in its largest orbit
    
\end{definition}

\begin{definition}[\textbf{Disjoint}]
    A cycle is \textbf{disjoint} if any integer is moved by at most one of these cycles
\end{definition}
\subsubsection{Even, Odd permutations}
\begin{definition}[\textbf{Transposition}]
    A cycle of length 2 is a \textbf{transposition}
\end{definition}

\begin{definition}[Even, Odd permutation]
    A permutation of a finite set is \textbf{even} or \textbf{odd} according to whether it can be expressed as a product of an even number of  transpositions or the product of an odd number of transpositions, respectively
\end{definition}

\begin{definition}[\textbf{Alternating Group}]
The subgroup of $S_n$ consisting of the even permutations of $n$ letters is the \textbf{alternating group $A_n$ on $n$ letters.}
\end{definition}


\end{multicols}

\end{document}
