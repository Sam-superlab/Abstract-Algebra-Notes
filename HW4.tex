\documentclass[12pt,letterpaper, onecolumn]{exam}
\usepackage{amsmath}
\usepackage{tikz}
\usepackage{tikz-cd}
\usetikzlibrary{positioning}
\usepackage{amsthm, amsfonts}
\usepackage{mathrsfs}

% Number equations by question (nice for homework)
\numberwithin{equation}{question}

% Theorem-like environments, numbered per question
\newtheorem{theorem}{Theorem}[question]
\newtheorem{lemma}[theorem]{Lemma}
\theoremstyle{definition}
\newtheorem{definition}[theorem]{Definition}
\theoremstyle{remark}
\newtheorem{remark}{Remark}

% Make the QED symbol a solid square (optional)
\renewcommand{\qedsymbol}{$\blacksquare$}
\newcommand{\Q}[0]{\mathbb{Q}}
\newcommand{\N}[0]{\mathbb{N}}
\newcommand{\Z}[0]{\mathbb{Z}}
\newcommand{\R}[0]{\mathbb{R}}
\newcommand{\G}[0]{\mathscr{G}}
\newcommand{\la}[0]{\langle}
\newcommand{\ra}{\rangle}
\newcommand{\mydistance}{.6cm}
\usepackage{amssymb}
\usepackage[lmargin=71pt, tmargin=1.2in]{geometry}  %For centering solution box
\lhead{ }
\rhead{ }
% \chead{\hline} % Un-comment to draw line below header
\thispagestyle{empty}   %For removing header/footer from page 1

\begin{document}

\begingroup  
    \centering
    \LARGE Abstract Algebra\\
    \LARGE Problem Set 4\\[0.5em]
    \large \today\\[0.5em]
    \large Xuyi (Sam) Ren\par
    % \large Roll Number\par
    \large MAT-321\par
\endgroup
\rule{\textwidth}{0.4pt}
\pointsdroppedatright   %Self-explanatory
\printanswers
\renewcommand{\solutiontitle}{\noindent\textbf{Ans:}\enspace}   %Replace "Ans:" with starting keyword in solution box
\begin{questions}

\question[4.29] Show that if $G$ is a finite group with identity $e$ and with an even number of elements, then there exists an element $a \neq e$ in $G$ such that $a * a = e$.
\begin{solution}

    Let $G$ be a group with even order $|G| = 2n$ for some $n \in \mathbb{N}$.
    Consider the set of non-identity elements $G \setminus \{e\}$.
    The cardinality of this set is:
    \[ |G \setminus \{e\}| = |G| - 1 = 2n - 1. \]
    Since $2n - 1$ is an odd number, $G \setminus \{e\}$ contains an odd number of elements.

    We prove by contradiction. Assume for contradiction that there is no element $a \in G \setminus \{e\}$ such that $a * a = e$.
    This assumption implies that for all $x \in G \setminus \{e\}$, $x \neq x^{-1}$.

    By Theorem 4.17, every element $x$ has exactly one inverse $x^{-1}$. Since $x \neq e$, we know $x^{-1} \neq e$, so $x^{-1} \in G \setminus \{e\}$.
    Because we assumed $x \neq x^{-1}$, we can pair every element $x$ in $G \setminus \{e\}$ with its unique and distinct inverse $x^{-1}$ to a subset $\{x, x^{-1}\}$ of size 2.
    
    These pairs are disjoint, they form a partition. Thus, $G \setminus \{e\}$ is the union of $k$ distinct pairs for some natural number $k$
    \[ G \setminus \{e\} = \{x_1, x_1^{-1}\} \cup \{x_2, x_2^{-1}\} \cup \dots \cup \{x_k, x_k^{-1}\} \]
    This implies that the total number of elements in $G \setminus \{e\}$ is $|G \setminus \{e\}| = 2k$, which is an even number.

    This contradicts our earlier observation that $|G \setminus \{e\}| = 2n - 1$ is an odd number.
    Therefore, our assumption is false. There must exist at least one element $a \in G \setminus \{e\}$ (so $a \neq e$) such that $a = a^{-1}$.
    
    Multiplying($*$) by $a$ on the right, we obtain
    \[ a * a = e. \]
\end{solution}



\question[4.37] Let $G$ be a group and suppose that $a * b * c = e$ for $a, b, c \in G$. Show that $b * c * a = e$ also.
\begin{solution}

    We are given $a * b * c = e$.
    Taking the inverse of both sides, we have:
    \[ (a * b * c)^{-1} = e^{-1} = e \]
    
    By Corollary 4.18, the inverse of a product is the product of the inverses in reverse order. Thus:
    \[ (a * b * c)^{-1} = c^{-1} * (a*b)^{-1}=c^{-1} * b^{-1} * a^{-1} \]
    together, we have
    \begin{equation}
        c^{-1} * b^{-1} * a^{-1} = e \label{eq:inverse_identity}
    \end{equation}

    Now, consider the expression we wish to evaluate: $b * c * a$.
    We can multiply the equation (\ref{eq:inverse_identity}) by $b * c$ on the left:
    \begin{align*}
        (b * c) * (c^{-1} * b^{-1} * a^{-1}) &= (b * c) * e \\
        b * (c * c^{-1}) * b^{-1} * a^{-1} &= b * c \quad (\text{By Associativity}) \\
        b * e * b^{-1} * a^{-1} &= b * c \quad (\text{Inverse   } c * c^{-1} = e) \\
        b * b^{-1} * a^{-1} &= b * c \quad (\text{Identity  }) \\
        e * a^{-1} &= b * c \quad (\text{Inverse  }) \\
        a^{-1} &= b * c
    \end{align*}
    
    Finally, we substitute this result ($b * c = a^{-1}$) back into our target expression $b * c * a$:
    \begin{align*}
        b * c * a &= (a^{-1}) * a \\
        &= e \quad (\text{Inverse  })
    \end{align*}
    
    Thus, $b * c * a = e$.
\end{solution}



\question[5.8] Determine whether the given set of invertible $n \times n$ matrices with real number entries is a subgroup of $GL(n,\mathbb{R})$: The $n \times n$ matrices with determinant $2$.
\begin{solution}
    Let $H$ be the set of $n \times n$ matrices with determinant 2:
    \[ H = \{ A \in GL(n, \mathbb{R}) \mid \det(A) = 2 \}. \]
    We claim that $H$ is not a subgroup of $GL(n, \mathbb{R})$.

    \begin{proof}
    We show that $H$ fails the closure requirement of Theorem 5.14 (1).
    Let $A$ and $B$ be arbitrary elements of $H$. By definition, $\det(A) = 2$ and $\det(B) = 2$.
    
    Recall from MAT215 the property of determinants for matrix multiplication:
    \[ \det(AB) = \det(A)\det(B). \]
    Substituting the values for $A$ and $B$:
    \[ \det(AB) = 2 \cdot 2 = 4. \]
    
    For $AB$ to be in $H$, we would require $\det(AB) = 2$. Since $4 \neq 2$, we have $AB \notin H$.
    
    Therefore, $H$ is not closed under matrix multiplication. By Theorem 5.14, $H$ is not a subgroup of $GL(n, \mathbb{R})$.
    \end{proof}
\end{solution}

In Exercises 41 and 42, let $\phi : G \to G'$ be an isomorphism of a group
$(G, *)$ with a group $(G', *')$. Write out a proof to convince a skeptic
of the intuitively clear statement.



\question[5.41]  If $H$ is a subgroup of $G$, then
\[
\phi[H] = \{\phi(h) \mid h \in H\}
\]
is a subgroup of $G'$. That is, an isomorphism carries subgroups into subgroups.
\begin{solution} 
Since $\phi$ is an injective on the entire domain $G$. Restricting the domain to a subset $H$ cannot break injectivity. Thus, $\phi_H$ is injective. By definition, $\phi$ maps $H$ onto its image $\phi[H]$, so it is surjective onto $\phi[H]$. Therefore, $\phi$ induces a bijection between $H$ and $\phi[H]$.


    Let $H \le G$. We claim that $\phi[H]$ is a subgroup of $G'$ by verifying the conditions of Theorem 5.14.

    \begin{proof}
    \textbf{1. Identity:}
    Since $H$ is a subgroup of $G$, the identity element $e$ of $G$ is in $H$.
    By the homomorphism property of $\phi$. $\phi(e) = e'$, where $e'$ is the identity of $G'$.
    Since $e \in H$, we have $\phi(e) \in \phi[H]$. Thus, $e' \in \phi[H]$.

    \textbf{2. Closure:}
    Let $x, y \in \phi[H]$ be arbitrary.
    By the definition of the image set $\phi[H]$, there exist $h_1, h_2 \in H$ such that $\phi(h_1) = x$ and $\phi(h_2) = y$.
    Consider the product $x *' y$:
    \begin{align*}
        x *' y &= \phi(h_1) *' \phi(h_2) \\
        &= \phi(h_1 * h_2) \quad (\text{Since } \phi \text{ is a homomorphism})
    \end{align*}
    Since $H$ is a subgroup, it is closed under the operation of $G$. Thus, $h_1 * h_2 \in H$.
    Consequently, $\phi(h_1 * h_2) \in \phi[H]$, so $x *' y \in \phi[H]$.

    \textbf{3. Inverses:}
    Let $x \in \phi[H]$. Then $x = \phi(h)$ for some $h \in H$.
    Since $H$ is a subgroup, $h^{-1} \in H$.
    Consider the inverse $x^{-1}$:
    \begin{align*}
        x^{-1} &= (\phi(h))^{-1} \\
        &= \phi(h^{-1}) \quad (\text{By Theorem 13.12b})
    \end{align*}
    Since $h^{-1} \in H$, it follows that $\phi(h^{-1}) \in \phi[H]$. Thus $x^{-1} \in \phi[H]$.

    Since $\phi[H]$ satisfies all conditions of Theorem 5.14, it is a subgroup of $G'$.
    \end{proof}




\end{solution}


\question[6.22] find all subgroups of the given group, and draw the subgroup diagram for the subgroups. $\Z_{12}$
\begin{solution}

    The group $\mathbb{Z}_{12} = \{0, 1, \dots, 11\}$ is a cyclic group of order 12 generated by 1.
    The divisors of 12 are $1, 2, 3, 4, 6, 12$.
    
    We compute the unique subgroup corresponding to each divisor (where the subgroup of order $k$ is generated by $12/k$):
    \begin{itemize}
        \item \textbf{Divisor 12 (Order 1):} $\langle 12/1 \rangle = \langle 12 \rangle = \langle 0 \rangle = \{0\}$
        \item \textbf{Divisor 6 (Order 2):} $\langle 12/2 \rangle = \langle 6 \rangle = \{0, 6\}$
        \item \textbf{Divisor 4 (Order 3):} $\langle 12/3 \rangle = \langle 4 \rangle = \{0, 4, 8\}$
        \item \textbf{Divisor 3 (Order 4):} $\langle 12/4 \rangle = \langle 3 \rangle = \{0, 3, 6, 9\}$
        \item \textbf{Divisor 2 (Order 6):} $\langle 12/6 \rangle = \langle 2 \rangle = \{0, 2, 4, 6, 8, 10\}$
        \item \textbf{Divisor 1 (Order 12):} $\langle 12/12 \rangle = \langle 1 \rangle = \mathbb{Z}_{12}$
    \end{itemize}

    We verify the containment relations. Note that $\langle a \rangle \subseteq \langle b \rangle$ if and only if $b$ divides $a$ (in $\mathbb{Z}_n$).
    \begin{itemize}
        \item $\langle 2 \rangle$ contains $\langle 4 \rangle$ and $\langle 6 \rangle$.
        \item $\langle 3 \rangle$ contains $\langle 6 \rangle$.
        \item $\langle 4 \rangle$ contains $\langle 0 \rangle$.
        \item $\langle 6 \rangle$ contains $\langle 0 \rangle$.
    \end{itemize}

    The lattice diagram is shown in Figure~\ref{fig:Z12-subgroups}.
\end{solution}


\begin{figure}[h]
\centering
\begin{tikzpicture}[node distance=1cm]

% Nodes
\node (Z12) {$\langle 1 \rangle = \mathbb{Z}_{12}$};

\node (Z6)  [below left=1cm and 1cm of Z12] {$\langle 2 \rangle$};
\node (Z4)  [below right=1cm and 1cm of Z12] {$\langle 3 \rangle$};

\node (Z3)  [below left=1cm and 1cm of Z6] {$\langle 4 \rangle$};
\node (Z2)  [below right=1cm and 1cm of Z4] {$\langle 6 \rangle$};

\node (Z1)  [below=1cm of Z3, xshift=3.5cm] {$\langle 12 \rangle = \{0\}$};

% Edges
\draw (Z12) -- (Z6);
\draw (Z12) -- (Z4);

\draw (Z6) -- (Z3);
\draw (Z6) -- (Z2);

\draw (Z4) -- (Z2);

\draw (Z3) -- (Z1);
\draw (Z2) -- (Z1);

\end{tikzpicture}
\caption{Subgroup of $\mathbb{Z}_{12}$}
\label{fig:Z12-subgroups}
\end{figure}


\question[] collaboration comments



\section{Appendix}

\end{questions}

\end{document}
