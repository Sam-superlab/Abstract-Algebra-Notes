\documentclass[11pt]{article}

% ---------- Page Layout ----------
\usepackage[
  paper=a4paper,
  left=0.8in,
  right=2.2in, % space for notes
  top=0.8in,
  bottom=0.8in
]{geometry}

% ---------- Math Packages ----------
\usepackage{amsmath, amssymb, amsthm, mathrsfs}

% ---------- Colors ----------
\usepackage{xcolor}
\usepackage{pagecolor}

\definecolor{paperCream}{RGB}{253,232,208}
\pagecolor{paperCream}

% ---------- Marginal Notes ----------
\setlength{\marginparpush}{10pt}
\usepackage{marginnote}
\renewcommand*{\marginfont}{\tiny}
\setlength{\marginparwidth}{1.7in}

% ---------- Formatting ----------
\setlength{\parindent}{0pt}
\setlength{\parskip}{6pt}

% ---------- tcolorbox Theorem/Proof Boxes (title bar like screenshot) ----------
\usepackage[most]{tcolorbox}

% Theme colors (tune these as you like)
\definecolor{thmBar}{RGB}{120,0,0}        % deep red title bar (like screenshot)
\definecolor{thmBody}{RGB}{255,248,244}   % very light warm body
\definecolor{thmFrame}{RGB}{120,0,0}      % frame matches title bar

\definecolor{prfBar}{RGB}{30,80,140}      % proof title bar (blue)
\definecolor{prfBody}{RGB}{248,249,251}   % proof body background
\definecolor{prfFrame}{RGB}{30,80,140}

% Base style for "bar-header" boxes
\tcbset{
  barbox/.style={
    enhanced,
    breakable,
    sharp corners,
    boxrule=0.8pt,
    left=8pt,right=8pt,top=6pt,bottom=6pt,
    before skip=10pt,
    after skip=10pt,
    coltitle=white,
    fonttitle=\bfseries,
    attach boxed title to top left={yshift=-1.6mm, xshift=3mm},
    boxed title style={
      sharp corners,
      boxrule=0pt,
      top=2pt,bottom=2pt,left=7pt,right=7pt
    },
  }
}

% Theorem-like environments (numbered within section)
\newtcbtheorem[number within=section]{theorem}{Theorem}{
  barbox,
  colback=thmBody,
  colframe=thmFrame,
  colbacktitle=thmBar,
  boxed title style={colback=thmBar}
}{thm}

\newtcbtheorem[use counter from=theorem]{lemma}{Lemma}{
  barbox,
  colback=thmBody,
  colframe=thmFrame,
  colbacktitle=thmBar,
  boxed title style={colback=thmBar}
}{lem}

\newtcbtheorem[use counter from=theorem]{proposition}{Proposition}{
  barbox,
  colback=thmBody,
  colframe=thmFrame,
  colbacktitle=thmBar,
  boxed title style={colback=thmBar}
}{prop}

\newtcbtheorem[use counter from=theorem]{corollary}{Corollary}{
  barbox,
  colback=thmBody,
  colframe=thmFrame,
  colbacktitle=thmBar,
  boxed title style={colback=thmBar}
}{cor}

% Proof box: wrap the amsthm proof environment (syntax unchanged)
\newtcolorbox{proofbox}{
  barbox,
  colback=prfBody,
  colframe=prfFrame,
  colbacktitle=prfBar,
  boxed title style={colback=prfBar},
  title=Proof
}

\let\oldproof\proof
\let\endoldproof\endproof
\renewenvironment{proof}{
  \begin{proofbox}\begin{oldproof}
}{
  \end{oldproof}\end{proofbox}
}

% ---------- Custom Note Command ----------
\renewcommand{\thmnote}[1]{\marginnote{\emph{#1}}}
\newcommand{\proofnote}[1]{\marginnote{\footnotesize\emph{#1}}}
\newcommand{\reason}[1]{\quad \text{(#1)}}

% ---------- Separators ----------
\newcommand{\sep}{%
  \vspace{0.6em}
  \noindent\rule{\linewidth}{0.4pt}
  \vspace{0.6em}
}

% ---------- Common Number Sets ----------
\newcommand{\R}{\mathbb{R}}
\newcommand{\N}{\mathbb{N}}
\newcommand{\Z}{\mathbb{Z}}
\newcommand{\Q}{\mathbb{Q}}
\newcommand{\G}{\mathscr{G}}
\newcommand{\bij}{\leftrightarrow}
% ---------- Document ----------
\begin{document}



\begin{center}
    {\Large \textbf{Theorem List w/ Proof}}\\
    \vspace{0.3em}
    {\small Abstract Algebra \quad • \quad Spring 2026 \quad • \quad Xuyi Ren}
\end{center}

% ==================================================

\begin{theorem*}
\textbf{3.13} Let $(S,\ast)$ be a binary algebraic structure.  
Then $(S,\ast)$ has at most one identity element.
\end{theorem*}

\begin{proof}
We proceed by contradiction.  
Assume that $e$ and $f$ are two distinct identity elements in $(S,\ast)$.

Since $e$ is an identity, we have
\[
e \ast x = x \quad \text{and} \quad x \ast e = x
\quad \text{for all } x \in S.
\]
In particular, taking $x=f$, we obtain
\[
e \ast f = f.
\]

Similarly, since $f$ is an identity, we have
\[
f \ast x = x \quad \text{and} \quad x \ast f = x
\quad \text{for all } x \in S.
\]
In particular, taking $x=e$, we obtain
\[
f \ast e = e.
\]

\proofnote{Use the defining property of an identity element, applied to the other candidate}

Now observe that
\[
e = f \ast e = f
\]
and hence $e=f$, which contradicts the assumption that $e$ and $f$ are distinct.

Therefore, $(S,\ast)$ has at most one identity element.
\end{proof}


\sep

\begin{theorem*}
\textbf{3.14} Let $(S,\ast)$ and $(S',\ast)$ be binary algebraic structures.
Let $\phi : S \to S'$ be an isomorphism.
Let $e \in S$ be an identity, and let $e' = \phi(e)$.
Then $e'$ is an identity element in $S'$.
\end{theorem*}

\begin{proof}
Let $y \in S'$ be arbitrary.
\proofnote{\textbf{Goal}: to prove $e'$ is an identity, we must show \textbf{$e' \ast y = y$ and $y \ast e' = y$ for arbitrary $y \in S'$}}

Since $\phi$ is surjective, there exists $x \in S$ such that
\[
\phi(x) = y.
\]
\proofnote{\textbf{Surjectivity} lets us \textbf{pull elements back} from $S'$ to $S$, where we know $e$ behaves as an identity}

Since $e$ is an identity element in $S$, we have
\[
e \ast x = x \quad \text{and} \quad x \ast e = x.
\]
\proofnote{\textbf{Identity in $S$}: this is where we actually use the hypothesis that $e$ is an identity}

Applying $\phi$ to both equations gives
\[
\phi(e \ast x) = \phi(x)
\quad \text{and} \quad
\phi(x \ast e) = \phi(x).
\]

Since $\phi$ is an isomorphism, it satisfies the homomorphism property:
\[
\phi(e \ast x) = \phi(e) \ast \phi(x)
\quad \text{and} \quad
\phi(x \ast e) = \phi(x) \ast \phi(e).
\]
\proofnote{\textbf{Homomorphism property}: structure is preserved under $\phi$}

Substituting, we obtain
\[
\phi(e) \ast \phi(x) = \phi(x)
\quad \text{and} \quad
\phi(x) \ast \phi(e) = \phi(x).
\]

Recalling that $\phi(e) = e'$ and $\phi(x) = y$, this becomes
\[
e' \ast y = y
\quad \text{and} \quad
y \ast e' = y.
\]

Since $y \in S'$ was arbitrary, $e'$ is an identity element in $S'$.
\proofnote{\textbf{Arbitrary element argument}: this completes the ``for all $y \in S'$'' requirement}
\end{proof}




\section*{Group theory}




\sep

\begin{theorem*}
\textbf{4.15 (Cancellation Law)} Let $(G,\ast)$ be a group, and let $a,b,c \in G$.
\begin{itemize}
    \item (Left cancellation) If $a \ast b = a \ast c$, then $b = c$.
    \item (Right cancellation) If $b \ast a = c \ast a$, then $b = c$.
\end{itemize}
\end{theorem*}

\begin{proof}
We prove the left cancellation law; the right cancellation law follows by a symmetric argument.

Assume that
\[
a \ast b = a \ast c.
\]
\proofnote{\textbf{Goal}: remove $a$ from the left without ``dividing''. \textbf{Key tool}: use $a^{-1}$ (\textbf{existence of inverses} is why cancellation holds in groups).}


Since $G$ is a group, $a$ has an inverse, denoted $a^{-1}$.
% \proofnote{Existence of inverses is essential; this is why cancellation holds in groups}

Multiply both sides on the left by $a^{-1}$ to obtain
\[
a^{-1} \ast (a \ast b) = a^{-1} \ast (a \ast c).
\]
\proofnote{\textbf{Left multiplication} replaces informal ``division''}

By associativity, we may rewrite both sides as
\[
(a^{-1} \ast a) \ast b = (a^{-1} \ast a) \ast c.
\]
\proofnote{\textbf{Associativity} justifies re-parenthesizing; this step is not automatic}

By definition of inverse, $a^{-1} \ast a = e$, where $e$ is the identity element of $G$.
Thus,
\[
e \ast b = e \ast c.
\]

By definition of identity, $e \ast b = b$ and $e \ast c = c$. Therefore,
\[
b = c,
\]
which completes the proof of left cancellation.
\end{proof}


\sep

\begin{theorem*}
\textbf{4.16} Let $(G,\ast)$ be a group, and let $a,b \in G$.
Then there exist unique elements $x,y \in G$ such that
\[
a \ast x = b
\quad \text{and} \quad
y \ast a = b.
\]
\end{theorem*}

\begin{proof}
We prove the statement for the equation $a \ast x = b$.
The proof for $y \ast a = b$ is analogous.

\proofnote{\textbf{Two tasks}: prove \textbf{existence} of a solution and then \textbf{uniqueness}}

\textbf{Existence.}
Since $G$ is a group, $a$ has an inverse $a^{-1}$.
Define
\[
x := a^{-1} \ast b.
\]
\proofnote{\textbf{Key idea}: ``solve'' by \textbf{multiplying on the left by the inverse}; no commutativity assumed}

Then
\[
\begin{aligned}
a \ast x
&= a \ast (a^{-1} \ast b) \\
&= (a \ast a^{-1}) \ast b \reason{associativity} \\
&= e \ast b \reason{definition of inverse} \\
&= b \reason{definition of identity}.
\end{aligned}
\]

Thus, a solution exists.

\medskip

\textbf{Uniqueness.}
Suppose $x'$ is another element of $G$ such that
\[
a \ast x' = b.
\]
Then
\[
a \ast x = b = a \ast x'.
\]
By the left cancellation law (Theorem 4.15), we conclude
\[
x = x'.
\]
\proofnote{\textbf{Cancellation} turns equality of products into equality of elements}

Therefore, the solution to $a \ast x = b$ is unique.
\end{proof}
\sep

\begin{lemma}
\textbf{3.A} Let $(S,\ast)$, $(S',\ast')$, and $(S'',\ast'')$ be binary algebraic structures.
\begin{enumerate}
    \item If $\phi : S \to S'$ is an isomorphism, then $\phi^{-1} : S' \to S$ is an isomorphism.
    \item If $\phi : S \to S'$ and $\psi : S' \to S''$ are isomorphisms, then
    $\psi \circ \phi : S \to S''$ is an isomorphism.
\end{enumerate}
\proofnote{\textbf{Key idea}: isomorphisms are closed under taking \textbf{inverses} and \textbf{composition}}
\end{lemma}

\begin{proof}
\textbf{(1)} Assume $\phi : S \to S'$ is an isomorphism.
Then $\phi$ is bijective, so $\phi^{-1} : S' \to S$ exists and is also bijective.

It remains to show that $\phi^{-1}$ preserves the operation.
Let $y_1,y_2 \in S'$ be arbitrary. Since $\phi$ is surjective, there exist $x_1,x_2 \in S$
such that $\phi(x_1)=y_1$ and $\phi(x_2)=y_2$.
Since $\phi$ is a homomorphism, we have
\[
\phi(x_1 \ast x_2)=\phi(x_1)\ast'\phi(x_2)=y_1\ast' y_2.
\]
Applying $\phi^{-1}$ to both sides yields
\[
x_1 \ast x_2=\phi^{-1}(y_1\ast' y_2).
\]
But $x_1=\phi^{-1}(y_1)$ and $x_2=\phi^{-1}(y_2)$, hence
\[
\phi^{-1}(y_1\ast' y_2)=\phi^{-1}(y_1)\ast \phi^{-1}(y_2).
\]
Therefore, $\phi^{-1}$ is a homomorphism, and since it is bijective, it is an isomorphism.

\medskip
\textbf{(2)} Assume $\phi : S \to S'$ and $\psi : S' \to S''$ are isomorphisms.
Then $\phi$ and $\psi$ are bijections, hence $\psi\circ\phi$ is a bijection.

It remains to show that $\psi\circ\phi$ preserves the operation.
Let $a,b \in S$ be arbitrary. Using the homomorphism property of $\phi$ and $\psi$, we get
\[
(\psi\circ\phi)(a\ast b)
= \psi(\phi(a\ast b))
= \psi(\phi(a)\ast'\phi(b))
= \psi(\phi(a))\ast''\psi(\phi(b))
 = (\psi\circ\phi)(a)\ast''(\psi\circ\phi)(b).
\]
Thus $\psi\circ\phi$ is a homomorphism. Since it is bijective, it is an isomorphism.
\end{proof}




\sep

\begin{theorem*}
\textbf{4.17} Let $(G,\ast)$ be a group.
Then there exists a unique element $e \in G$ such that
\[
e \ast x = x \ast e = x
\quad \text{for all } x \in G.
\]
Moreover, for each $a \in G$, there exists a unique element $a' \in G$ such that
\[
a' \ast a = a \ast a' = e.
\]
In particular, the identity element and the inverse of each element are unique in a group.
\end{theorem*}

\begin{proof}
    The first statement is by $\G 2$ we just need to show uniqueness by Theorem 3.13
    \\
    The second statement existence by $\G 3$, we just need to show uniquesness:
    Suppose we have $a',a''\in G$ s.t. 
    \begin{align}
        a*a'=&e=a'*a\\
        a*a''=&e=a''*a
    \end{align}
    Then we have $a''*a=a'*a$ by cancellation law, we have $a''=a'$
\end{proof}

\sep

\begin{corollary}
\textbf{4.18} Let $(G,\ast)$ be a group, and let $a,b \in G$. Then
\[
(a\ast b)'=b'\ast a'.
\]
\end{corollary}

\begin{proof}
We show that $b'\ast a'$ is an inverse of $a\ast b$. By uniqueness of inverses (Theorem~4.17), this will imply
\((a\ast b)'=b'\ast a'\).

First, compute one direction:
\[
\begin{aligned}
(b'\ast a')\ast (a\ast b)
&= \bigl((b'\ast a')\ast a\bigr)\ast b
\quad \text{(associativity)}\\
&= \bigl(b'\ast (a'\ast a)\bigr)\ast b
\quad \text{(associativity)}\\
&= (b'\ast e)\ast b
\quad \text{(inverse property)}\\
&= b'\ast b
\quad \text{(identity)}\\
&= e.
\quad \text{(inverse property)}
\end{aligned}
\]

Similarly, for the other direction:
\[
\begin{aligned}
(a\ast b)\ast (b'\ast a')
&= a\ast \bigl(b\ast (b'\ast a')\bigr)
\quad \text{(associativity)}\\
&= a\ast \bigl((b\ast b')\ast a'\bigr)
\quad \text{(associativity)}\\
&= a\ast (e\ast a')
\quad \text{(inverse property)}\\
&= a\ast a'
\quad \text{(identity)}\\
&= e.
\quad \text{(inverse property)}
\end{aligned}
\]

Thus $b'\ast a'$ is a two-sided inverse of $a\ast b$, so by Theorem~4.17 it must equal $(a\ast b)'$.
\end{proof}


\sep



\begin{theorem*}
\textbf{5.14 (Subgroup Criterion)} Let $(G,\ast)$ be a group, and let $H \subseteq G$.
Then $H$ is a subgroup of $G$ if and only if the following three conditions all hold:\proofnote{
\textbf{Key idea}: associativity is \textbf{inherited} from $G$ and never needs to be re-verified.\\
\textbf{Practical use}: this criterion replaces checking all group axioms directly.
}
\begin{enumerate}
    \item $H$ is \textbf{closed} under the binary operation of $G$;
    \item the \textbf{identity element} $e$ of $G$ is contained in $H$;
    \item for every $a \in H$, the \textbf{inverse} $a^{-1}$ is also in $H$.
\end{enumerate}


\end{theorem*}




\sep

\begin{lemma}
\textbf{5B (One-Step Subgroup Test)} Let $(G,\ast)$ be a group, and let $H$ be a nonempty subset of $G$.
Then $H \le G$ if and only if for all $x,y \in H$, we have
\[
x \ast y^{-1} \in H.
\]
\proofnote{\textbf{One-step test}: instead of three conditions, check closure under $x y^{-1}$}
\end{lemma}

\begin{proof}
We prove both directions.

\medskip
\textbf{($\Rightarrow$)}
Assume that $H$ is a subgroup of $G$.
Let $x,y \in H$ be arbitrary.
Since $H$ is a subgroup, $y^{-1} \in H$ by Theorem~5.14, and since $H$ is closed under the group operation,
\[
x \ast y^{-1} \in H.
\]
\proofnote{\textbf{Easy direction}: closure and inverses are already known for subgroups}

\medskip
\textbf{($\Leftarrow$)}
Now assume that $H$ is nonempty and that for all $x,y \in H$,
\[
x \ast y^{-1} \in H.
\]
We will verify the three conditions of Theorem~5.14.

\proofnote{\textbf{Strategy}: prove the subgroup axioms \textbf{out of order}—identity, inverses, then closure}

\medskip
\textbf{Identity.}
Since $H$ is nonempty, choose some element $x \in H$.
Let $y = x$. Then
\[
x \ast x^{-1} = e \in H,
\]
so the identity element $e$ of $G$ lies in $H$.
\proofnote{\textbf{Nonemptiness} is essential here; otherwise $x$ cannot be chosen}

\medskip
\textbf{Inverses.}
Let $y \in H$ be arbitrary.
Since $e \in H$ from the previous step, applying the hypothesis with $x=e$ gives
\[
e \ast y^{-1} = y^{-1} \in H.
\]

\medskip
\textbf{Closure.}
Let $x,y \in H$.
From the previous step, $y^{-1} \in H$, and applying the hypothesis to $x$ and $y^{-1}$ gives
\[
x \ast (y^{-1})^{-1} \in H.
\]
Since $(y^{-1})^{-1} = y$, we conclude
\[
x \ast y \in H.
\]

Thus $H$ contains the identity, is closed under inverses, and is closed under the group operation.
By Theorem~5.14, $H$ is a subgroup of $G$.
\end{proof}


% =======================
% (Paste into your doc)
% =======================

\sep

\begin{theorem*}
\textbf{5.17} Let $G$ be a group and let $a \in G$. Then
\[
H=\{a^n \mid n \in \Z\}
\]
is a subgroup of $G$ and is the smallest subgroup of $G$ that contains $a$; that is,
every subgroup containing $a$ contains $H$.
\end{theorem*}

\begin{proof}
For the first statement we prove by using lemma 5b. 

Notice that there exits $x'\in\langle x \rangle$ and $x'=x$. Thus $\langle x\rangle\ne \emptyset$. Use Lemma 5b, let $a,b\in \langle x \rangle$ be arbitrary such that we can fix a choice of $m,n\in \Z$ s.t. $a=x^m,b=x^n$, then
\begin{align*}
    ab^{-1}&=x^m(x^n)^{-1}\\
    &=x^mx^{-n}\\
    &=x^{(m-n)}
\end{align*}
By lemma 5A. We know that $m-n\in \Z$ so $ab^{-1}\in \langle x\rangle$ as needed.

The second statement:

Now suppose $H\le G$ and $x\in H$ we wish to show that $\langle x\rangle \le H$. We know $x'=x\in H$, $x^0=e\in H$  and $x^{-1}\in H$. By 5.13, we could prove by induction that both $x^n,x^{-n}\in H$ for arbitrary $n\in \Z$.
\end{proof}

\sep

\begin{lemma}
\textbf{6.A (Division Algorithm)} If $m \in \Z^{+}$ and $n \in \Z$, then there exist unique integers $q,r$ such that
\[
n=mq+r \quad \text{and} \quad 0 \le r < m.
\]
\end{lemma}
\begin{proof}
(TODO)
\end{proof}

\sep

\begin{theorem*}
\textbf{6.1} Every cyclic group is abelian.
\end{theorem*}
\begin{proof}
(TODO)
\end{proof}

\sep

\begin{theorem*}
\textbf{6.6} A subgroup of a cyclic group is cyclic.
\end{theorem*}
\begin{proof}
(TODO)
\end{proof}

\sep

\begin{corollary}
\textbf{6.7} The subgroups of $(\Z,+)$ are precisely the groups $(n\Z,+)$ for $n \in \N$.
\end{corollary}
\begin{proof}
(TODO)
\end{proof}

\sep

\begin{theorem*}
\textbf{6.10} Let $G$ be a cyclic group with generator $a$.
If the order of $G$ is infinite, then $G$ is isomorphic to $(\Z,+)$.
If $G$ has finite order $n$, then $G$ is isomorphic to $(\Z_n,+_n)$.
\end{theorem*}
\begin{proof}
(TODO)
\end{proof}

\sep

\begin{lemma}
\textbf{6.B} Given $r,s,t \in \Z$, if $\gcd(r,s)=1$ and $r$ divides $st$, then $r$ divides $t$.
\end{lemma}


\begin{lemma}
\textbf{6.C} For $n,s \in \Z^{+}$, if $g=\gcd(n,s)$, then $\gcd(n/g,s/g)=1$.
\end{lemma}

\sep

\begin{lemma}
\textbf{6.D} If $G$ is a finite cyclic group with generator $x$, then the order of $G$ is the smallest
$m \in \Z^{+}$ such that $x^{m}=e$.
For $n \in \Z$, we have $x^{n}=e$ if and only if the order of $G$ divides $n$.
\end{lemma}
\begin{proof}
(TODO)
\end{proof}

\sep

\begin{theorem*}
\textbf{7.4} The intersection of a family of subgroups $H_i$ of a group $G$ (for $i \in I$) is again a subgroup of $G$.
\end{theorem*}
\begin{proof}
(TODO)
\end{proof}

\sep

\begin{lemma}
\textbf{8.A} Let $\phi : G \to G'$ be a homomorphism.
Let $e \in G$ and $e' \in G'$ be the identity elements. Then:
\begin{enumerate}
    \item $\phi(e)=e'$;
    \item for any $x \in G$, $\phi(x^{-1})=\phi(x)^{-1}$;
    \item if $H \le G$, then $\phi[H] \le G'$.
\end{enumerate}
\end{lemma}
\begin{proof}
Sorry
\end{proof}

\sep


\subsubsection*{8.15, The lemma we use to prove Cayley's Theorem}
\begin{lemma*}


Let $G$ and $G'$ be groups and let $\phi : G \to G'$ be a one-to-one function such that
\[
\phi(xy)=\phi(x)\phi(y) \quad \text{for all } x,y \in G.
\]
Then $\phi[G]$ is a subgroup of $G'$ and $\phi$ provides an isomorphism of $G$ with $\phi[G]$.
\end{lemma*}
\begin{proof}
We wish to prove by theorem 5.14. First, we prove $\phi[G]$ is closed under $\phi$.

Let $x',y'\in \phi[G]$ be arbitrary, then we can fix a choice of $x,y\in G$ s.t. $\phi(x)=x',\phi(y)=y'$. Apply homomorphism property we have \begin{equation*}
    \phi(xy)=\phi(x)\phi(y) =x'y' \in \phi[G]
\end{equation*}
we haven shown the closure.

Now we show the identity property for subgroup.

let $e'\in G'$ be identity, then we have \proofnote{Here we want the identity of $G'$ in $\phi[G]$, we prove this by using similar idea of proving the identity to be unique, and consider the element $\phi(e)$}
\begin{equation*}
    e'\phi(e)=\phi(e)=\phi(ee)=\phi(e)\phi(e)
\end{equation*}
applying the cancellation law from left, we have $e'=\phi(e)$.

Now we show the inverse property of subgroup

Let $x'\in \phi[G]$ be arbitrary, we can fix a choice of $x\in G$ s.t. $\phi(x)=x'$. 

consider $e'$ 
\begin{equation*}
    e'=\phi(e)=\phi(xx^{-1})=\phi(x)\phi(x^{-1})=x'\phi(x^{-1})
\end{equation*}
which shows that $x'^{-1}=\phi(x^{-1})\in \phi[G]$. we have shown that $\phi[G]\le G'$.

Is clear that $\phi:G\bij \phi[G]$ and the homomorphism holds. Thus indeed $\phi$ is isomorphism.
\end{proof}


\subsubsection{8.16, Cayley's Theorem}

\begin{theorem*}
\textbf{8.16 (Cayley’s Theorem)} Every group is isomorphic to a group of permutations.
\end{theorem*}
\begin{proof}

\end{proof}

\sep

\begin{theorem*}
\textbf{9.8} Every permutation $\sigma$ of a finite set is a product of disjoint cycles.
\end{theorem*}
\begin{proof}
Sorry
\end{proof}

\sep

\begin{corollary}
\textbf{9.12} Any permutation of a finite set of at least two elements is a product of transpositions.
\end{corollary}
\begin{proof}
Sorry
\end{proof}

\section*{Ring Theory}




\section{Additional Theorems}

\begin{theorem*}
    \textbf{8.5} $S_A$, the set of all permutations of $A$ forms a group under permutation multiplication.
\end{theorem*}
\begin{proof}
    sorry
\end{proof}

\begin{theorem*}
    \textbf{Subgroup of Cyclic group} can be \textbf{unequally} determined by the divisors of $n$
\end{theorem*}

\begin{proof}
    sorry
\end{proof}

\begin{theorem*}
(\textbf{Abelian condition})    If $H\le G$ and $G$ is abelian, then $gHg^{-1}=H$ for all $g\in G$
\end{theorem*}

\begin{proof}
    sorry
\end{proof}
\end{document}
