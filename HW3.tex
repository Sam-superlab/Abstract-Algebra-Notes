\documentclass[12pt,letterpaper, onecolumn]{exam}
\usepackage{amsmath}
\usepackage{tikz}
\usepackage{tikz-cd}
\usepackage{amsthm}
\usepackage{mathrsfs}

% Number equations by question (nice for homework)
\numberwithin{equation}{question}

% Theorem-like environments, numbered per question
\newtheorem{theorem}{Theorem}[question]
\newtheorem{lemma}[theorem]{Lemma}
\theoremstyle{definition}
\newtheorem{definition}[theorem]{Definition}
\theoremstyle{remark}
\newtheorem{remark}{Remark}

% Make the QED symbol a solid square (optional)
\renewcommand{\qedsymbol}{$\blacksquare$}
\newcommand{\Q}[0]{\mathbb{Q}}
\newcommand{\N}[0]{\mathbb{N}}
\newcommand{\Z}[0]{\mathbb{Z}}
\newcommand{\R}[0]{\mathbb{R}}
\usepackage{amssymb}
\usepackage[lmargin=71pt, tmargin=1.2in]{geometry}  %For centering solution box
\lhead{ }
\rhead{ }
% \chead{\hline} % Un-comment to draw line below header
\thispagestyle{empty}   %For removing header/footer from page 1

\begin{document}

\begingroup  
    \centering
    \LARGE Abstract Algebra\\
    \LARGE Problem Set 1\\[0.5em]
    \large \today\\[0.5em]
    \large Xuyi (Sam) Ren\par
    % \large Roll Number\par
    \large MAT-321\par
\endgroup
\rule{\textwidth}{0.4pt}
\pointsdroppedatright   %Self-explanatory
\printanswers
\renewcommand{\solutiontitle}{\noindent\textbf{Ans:}\enspace}   %Replace "Ans:" with starting keyword in solution box
\begin{questions}

\question[] 4.8 The set $\{1,3,5,7\}$ under $\cdot_8$ modulo 8 is a group. Give the table of the group

\begin{solution}
 {General Idea:}
To construct the Cayley table for $G=\{1,3,5,7\}$ under multiplication modulo 8, we compute $a \cdot b \pmod 8$ for all pairs. We anticipate that 1 will be the identity element. Since multiplication of integers is commutative, we expect the table to be symmetric across the main diagonal. We will explicitly check that every row and column contains each element exactly once, which is characteristic of group tables.

\begin{center}
\begin{tabular}{c|cccc}
$\cdot_8$ & 1 & 3 & 5 & 7 \\
\hline
1 & 1 & 3 & 5 & 7 \\
3 & 3 & 1 & 7 & 5 \\
5 & 5 & 7 & 1 & 3 \\
7 & 7 & 5 & 3 & 1 \\
\end{tabular}
\end{center}

\end{solution}



\question[] All $n\times n$ diagonal matrices with non-zero diagonal entries form a group under matrix multiplication.

\begin{solution}
 {General Idea:}
To prove this, we must first rigorously define the set of diagonal matrices $D^*$. A matrix $A$ is diagonal if $A_{ij} = 0$ whenever $i \neq j$. We will use the standard definition of matrix multiplication, $(AB)_{ij} = \sum_{k=1}^n A_{ik} B_{kj}$, to show that the product of two diagonal matrices is still diagonal (Closure). This summation definition also allows us to derive the specific values of the diagonal entries, which will simplify finding the Identity and Inverse elements.


    Let $D^*$ be the set of all $n \times n$ diagonal matrices with non-zero entries on the main diagonal. Formally, $A \in D^*$ if:
    \begin{itemize}
        \item $A_{ij} = 0$ for all $i \neq j$.
        \item $A_{ii} \in \mathbb{R} \setminus \{0\}$ for all $1 \leq i \leq n$.
    \end{itemize}

    We first verify if matrix multiplication is binary operation under diagonal matrices(except for the zero matrix)


     {1. Closure}
    Let $A, B \in D^*$. By the definition of matrix multiplication, the $(i,j)$-th entry of the product $C = AB$ is
    \[ C_{ij} = \sum_{k=1}^n A_{ik} B_{kj} \]
    
    Since $A$ is diagonal, $A_{ik} = 0$ unless $k=i$. Thus, the summation over $k$ only sum to a single term where $k=i$:
    \[ C_{ij} = A_{ii} B_{ij} \]
    
    Now consider the term $B_{ij}$. Since $B$ is diagonal, $B_{ij} = 0$ unless $j=i$.
    \begin{itemize}
        \item \textbf{Case 1 ($i \neq j$):} Since $j \neq i$, $B_{ij} = 0$. Therefore, $C_{ij} = A_{ii} \cdot 0 = 0$. This confirms $C$ is a diagonal matrix.
        \item \textbf{Case 2 ($i = j$):} The entry is $C_{ii} = A_{ii} B_{ii}$. Since $A_{ii} \neq 0$ and $B_{ii} \neq 0$, their product $C_{ii} \neq 0$.
    \end{itemize}
    Thus, $AB \in D^*$, and the operation is closed.

We verify the group axioms for $\langle D^*, \cdot \rangle$

     {2. Associativity}
    Matrix multiplication is associative for all $n \times n$ matrices (by MAT215). Since $D^* \subset M_{n \times n}(\mathbb{R})$, the induced operation keeps associativity.

     {3. Identity}
    Consider the identity matrix $I_n$ where $(I_n)_{ij} = \delta_{ij}$ (the Kronecker delta).
    For any $A \in D^*$, the $(i,j)$-th entry of $A I_n$ is:
    \[ (A I_n)_{ij} = \sum_{k=1}^n A_{ik} \delta_{kj} \]
    The term $\delta_{kj}$ is non-zero only when $k=j$. Thus the sum collapses to $A_{ij} \cdot 1 = A_{ij}$.
    Similarly, $(I_n A)_{ij} = A_{ij}$.
    Since $A_{ii} \neq 0$ implies the diagonal entries of $I_n$ (all 1s) are non-zero, $I_n \in D^*$. Thus, $I_n$ is the identity.

     {4. Inverse}
    Let $A \in D^*$. We seek a matrix $B \in D^*$ such that $(AB)_{ij} = \delta_{ij}$.
    Using our derived formula for the diagonal product from the Closure section, we know $(AB)_{ii} = A_{ii} B_{ii}$.
    We require:
    \[ A_{ii} B_{ii} = 1 \implies B_{ii} = \frac{1}{A_{ii}} \]
    Since $A_{ii} \neq 0$, the value $1/A_{ii}$ is defined and non-zero. Let $B$ be the diagonal matrix with diagonal entries $B_{ii} = 1/A_{ii}$.
    Then:
    \[ (AB)_{ij} = \begin{cases} 0 & i \neq j \\ A_{ii}(1/A_{ii}) = 1 & i = j \end{cases} \]
    This gives the identity matrix $I_n$. By Theorem 4.17, inverses in a group are unique, so $B = A^{-1}$.

    Therefore, $D^*$ is a group.
\end{solution}



\question[] Let $S$ be the set of all real numbers except -1, deine $*$ on $S$ by 
\begin{equation*}
    a*b=a+b+ab
\end{equation*}
\begin{enumerate}
    \item Show that * gives a binary operation on $S$
    \item Show that $<S,*>$ is a group
    \item Find the solution of the equation $2*x*3=7$ in $S$
\end{enumerate}
\begin{solution}
 {General Ideas:}
The operation $a*b = a+b+ab$ can be factored as $(a+1)(b+1) - 1$. This insight reveals that the operation $*$ on the set $S = \mathbb{R} \setminus \{-1\}$ is isomorphic to standard multiplication on $\mathbb{R} \setminus \{0\}$ by the shift 1 unit $x \to x+1$.

Closure \& Inverse: The condition $x \neq -1$ is equivalent to $x+1 \neq 0$. Since the product of non-zero numbers is non-zero, closure and inverses follow naturally from this shifted perspective.

Equation Solving: For the equation $2*x*3=7$, we rely on Theorem 4.16 (uniqueness of solutions in a group) to guarantee a single valid $x$. We then expand the operation algebraically to isolate $x$.



First, we show that * gives a binary operation on S. 
    A Binary Operation * on set $S$ is a function $*:S\times S\to S$, For each $(a,b)\in S\times S$ we denote $*((a,b))$ as $a*b$
    \begin{proof}
        Let $(x,y)\in S\times S$ be arbitrary. We wish to show that the mapping $*$ is a function by showing:
        \begin{itemize}
            \item The existence of output
            \item The uniqueness of output under same input
        \end{itemize}

First we prove the existence condition. That is, there exists an element $x*y$ in $S$.

Since $x,y \in S$, both $x$ and $y$ are real numbers that does not equal $-1$.  
By definition of $*$, we define
\[
x*y := x + y + xy .
\]
The expression $x+y+xy$ is a real number because $\mathbb{R}$ is closed under addition and multiplication.  
Hence, for every $(x,y)\in S\times S$, the value $x*y$ exists as a real number.

Next, we prove the uniqueness condition.  
Suppose that for the same input $(x,y)\in S\times S$, there exist two outputs $u,v\in S$ such that
\[
u = x*y \quad \text{and} \quad v = x*y .
\]
By substitution using the definition of $*$, we have
\[
u = x+y+xy = v .
\]
Thus $u=v$, which proves that the output corresponding to $(x,y)$ is unique.

Therefore, the operation $*$ defines a function from $S\times S$ to $\mathbb{R}$.
    \end{proof}






Now we show that $*$ gives a closed binary operation on $S$
\begin{proof}
    Let $x,y\in S$ be arbitrary. Since $S = \mathbb{R} \setminus \{-1\}$, we know $x \neq -1$ and $y \neq -1$.
    This implies $x+1 \neq 0$ and $y+1 \neq 0$.
    Since the product of two non-zero real numbers is non-zero:
    \begin{equation*}
        (x+1)(y+1) \neq 0
    \end{equation*}
    Expanding this, we get $xy + x + y + 1 \neq 0$, which implies:
    \begin{equation*}
        x + y + xy \neq -1
    \end{equation*}
    Thus, that is exactly $x*y \neq -1$, therefore $x*y \in S$. The operation is closed.
\end{proof}

Now we show that $<S,*>$ is a group by showing associativity, identity, inverse

Associativity:
\begin{proof}
    We wish to prove the associativity of $*$ on $S$

    Let $x,y,z\in S$ be arbitrary. Consider the expression:
    \begin{align*}
        x*(y*z)&=x*(y+z+yz)\\
        &= x+y+z+yz+x(y+z+yz) \\
         &= x+y+z+yz+xy+xz+xyz
    \end{align*}
Now we consider another expression:
\begin{align*}
    (x*y)*z&=(x+y+xy)*z\\
    &=x+y+z+xy+z(x+y+zy)\\
    &=x+y+z+xy+xz+yz+xyz
\end{align*}
Comparing each term in both expression, we notice that they have the same expanded form, thus we have $x*(y*z)=(x*y)*z$ for all $x,y,z\in S$
    
\end{proof}


Identity:
\begin{proof}
    We now prove there exists an identity element.
    Let $x\in S$ be arbitrary, consider real number 0, we claim that 0 is the identity element of $<S,*>$.
    We shall verify this by checking both $x*0$ and $0*x$:
    \begin{align*}
        x*0&=x+0+x\cdot0=x\\
        0*x&=0+x+0\cdot x=x
    \end{align*}
    Thus, we have $x*0=0*x=x$ for arbitrary $x$, therefore 0 is indeed an identity element of $<S,*>$
\end{proof}

Inverse:\begin{proof}
    Let $x \in S$ be arbitrary. We are looking for an element $x' \in S$ such that $x * x' = 0$ (the identity we found in previous proof).
    \begin{align*}
        x + x' + xx' &= 0 \\
        x + x'(1+x) &= 0 \\
        x'(1+x) &= -x
    \end{align*}
    Since $x \in S$, we know $x \neq -1$, so $1+x \neq 0$. Thus we can divide by $1+x$:
    \begin{equation*}
        x' = -\frac{x}{1+x}
    \end{equation*}
    We must verify that $x' \in S$, namely, $x'\neq -1$. Assume for the sake of contradiction that $x' = -1$.
    \begin{align*}
        -1 &= -\frac{x}{1+x} \\
        1+x &= x \\
        1 &= 0
    \end{align*}
    This is a contradiction. Therefore, $x' \neq -1$, so $x' \in S$. Every element has an inverse correspond to it.
\end{proof}



We seek the solution $x \in S$ for the equation $2*x*3=7$.
Since we proved in Part 2 that $< S, * >$ is a group, Theorem 4.16  guarantees that linear equations of the form $a*y=b$ and $y*a=b$ have unique solutions in $G$. By applying this theorem twice, we know a unique solution $x$ exists.

We solve for $x$ by evaluating the LHS using the definition of the binary operation $a*b = a+b+ab$:

First, we evaluate the $2*x$
\[ 2*x = 2 + x + 2(x) = 2 + 3x  \]

Next, we evaluate $(2*x)*3$ by substituting $(2+3x)$ for the first term
\begin{align*}
    (2*x)*3 &= (2+3x)*3 \\
    &= (2+3x) + 3 + (2+3x)(3) \quad (\text{Definition of } *) \\
    &= 2 + 3x + 3 + 6 + 9x \\
    &= 12x + 11
\end{align*}

We equate this result to the right-hand side given in the problem
\begin{align*}
    12x + 11 &= 7 \\
    12x &= -4 \\
    x &= -\frac{4}{12} = -\frac{1}{3}
\end{align*}

Finally, we verify that this solution lies in $S$. The set is defined as $S = \mathbb{R} \setminus \{-1\}$. Since $-\frac{1}{3} \neq -1$, we have $-\frac{1}{3} \in S$.

Thus, the unique solution is $x = -\frac{1}{3}$.




\end{solution}


\question[] 24. Give a table for a binary operation on the set $\{e, a, b\}$ of three elements satisfying axioms $\mathscr{G}_2$ and $\mathscr{G}_3$ for a group but not axiom $\mathscr{G}_1$.
\begin{solution}

    Let $S = \{e, a, b\}$. We define the binary operation $*$ on $S$ by the following table:

    \begin{center}
    \begin{tabular}{c|ccc}
         $*$ & $e$ & $a$ & $b$ \\
         \hline
         $e$ & $e$ & $a$ & $b$ \\
         $a$ & $a$ & $e$ & $b$ \\
         $b$ & $b$ & $a$ & $e$
    \end{tabular}
    \end{center}

    We now verify the axioms:

     {1.  $\mathscr{G}_2$ (Identity)}\\
    Observing the first row and first column of the table, we see that for all $x \in S$:
    \[ e*x = x \quad \text{and} \quad x*e = x \]
    Thus, $e$ is the identity element.

     {2.  $\mathscr{G}_3$ (Inverses)}\\
    We check the diagonal entries to find the inverse for each element:
    \begin{itemize}
        \item $e*e = e \implies e^{-1} = e$
        \item $a*a = e \implies a^{-1} = a$
        \item $b*b = e \implies b^{-1} = b$
    \end{itemize}
    Since every element in $S$ has an inverse in $S$, axiom $\mathscr{G}_3$ is satisfied.

     {3. $\mathscr{G}_1$ (Associativity) does not hold}\\
    To show the operation is not associative, we provide a specific counterexample using elements $a$ and $b$.
    
    Consider the expression $(a*b)*a$:
    \begin{align*}
        (a*b)*a &= (b)*a \quad (\text{From table, } a*b=b) \\
        &= a \quad \quad \quad \ (\text{From table, } b*a=a)
    \end{align*}
    
    Now consider the expression $a*(b*a)$:
    \begin{align*}
        a*(b*a) &= a*(a) \quad (\text{From table, } b*a=a) \\
        &= e \quad \quad \quad \ (\text{From table, } a*a=e)
    \end{align*}
    
    Since $(a*b)*a = a$ and $a*(b*a) = e$, and $a \neq e$, we have:
    \[ (a*b)*a \neq a*(b*a) \]
    Therefore, the operation is not associative.
\end{solution}

\question[] 32. Show that every group $G$ with identity $e$ and such that $x*x=e$ for all $x \in G$ is abelian. 
\begin{solution}
\paragraph{General Idea:}

We are given a group where every element is its own inverse (i.e., $x*x=e$ implies $x = x^{-1}$). We want to prove the group is abelian, meaning $a*b = b*a$ for all $a, b$.

The strategy is to start with the identity element in a specific form, by hint, consider $(a*b)*(a*b)=e$, and then use the group properties (associativity and cancellation) to isolate $a*b$ and $b*a$ on opposite sides of the equation.



Let $a, b \in G$ be arbitrary elements.

We are given the property that $x*x = e$ for all $x \in G$. Applying this to the element $(a*b) \in G$, we have:

\begin{equation}
(ab)(a*b) = e \label{eq:self_inverse}
\end{equation}

We star($*$) both sides of equation (\ref{eq:self_inverse}) by $a$ on the left:

\begin{align*}
    a * ((a*b)*(a*b)) &= a * e \\
    a * (a*(b*(a*b))) &= a \quad (\text{By Associativity on } a, (a*b), (a*b) \text{ and Identity property}) \\
    (a*a) * (b*(a*b)) &= a \quad (\text{By Associativity on } a, a, (b*(a*b))) \\
    e * (b*(a*b)) &= a \quad (\text{Since } a*a=e \text{ by given hypothesis}) \\
    b * (a*b) &= a \quad (\text{By Identity property})
\end{align*}



Now, we star this result by $b$ on the left:

\begin{align*}
    b * (b * (a*b)) &= b * a \\
    (b*b) * (a*b) &= b * a \quad (\text{By Associativity on } b, b, (a*b)) \\
    e * (a*b) &= b * a \quad (\text{Since } b*b=e \text{ by given hypothesis}) \\
    a*b &= b*a \quad (\text{By Identity property})
\end{align*}



Since $a*b = b*a$ for all arbitrary $a, b \in G$, the group $G$ is abelian.

\end{solution}

\section{Appendix}

\end{questions}

\end{document}
